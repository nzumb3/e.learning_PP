\documentclass[noindent]{tudreport}
\usepackage[utf8]{inputenc}
\usepackage[german]{babel}
\usepackage{amsmath}
\usepackage{listings}
\title{TUDas - Organisationsapp der Technischen Universität Darmstadt}
\subtitle{Ergebnis des E-Learning Projektpraktikums WS18-19}
\subsubtitle{Benedikt Lins (1799381) und Stefan Thaut (1800351)\\
			Fachbereich 20 - Informatik\\
			\today}
\setlength{\parindent}{0em} 
\begin{document}
	\lstset{language=Java}
	\maketitle
	\tableofcontents
	
	\chapter{Motivation}
		Jeder Student oder der, der einmal einer gewesen ist, kennt es: Du kommst das erste Mal in die Uni und hast keine Ahnung, was zu tun ist. Alles ist komplett neu und Du bist froh, wenn Du die Räume findest, in denen Du Dich laut Deinem Willkommensbrief einfinden sollst. Selbst wenn Du dann nach ein paar Tagen herausgefunden hast, welche Straßenbahn Dich in die Uni bringt, bist Du nach ein paar Wochen immer noch ratlos, welche Dokumente du gegebenenfalls wo nachreichen musst, wo Du Dich zu welchen Veranstaltungen und Prüfungen anmelden sollst und in welchem Kellerraum nun dieser Treffpunkt Mathe stattfindet.\\
		
		Diesem Problem soll sich die (Android-)App \textit{TUDas} widmen. Die App soll als Organisationsplattform für Studenten verschiedener Fachbereiche während ihres ersten Semesters dienen. Dabei soll TUDas vorhandene Plattformen, wie \textit{Moodle}\footnote{www.moodle.tu-darmstadt.de [zuletzt aufgerufen: 23.11.2018]} oder die \textit{OAPP} \footnote{www.oapp.tu-darmstadt.de [zuletzt aufgerufen: 23.11.2018]} unterstützen und nicht ersetzen.\\
		%TODO worin unterscheidet sich die App von OAPP und Moodle und Google Maps etc?
		
		Diese Dokumentation soll einerseits die Funktionalitäten der App TUDas festhalten und andererseits als Übersicht für zukünftige Entwickler dienen. Kapitel \ref{} behandelt die technischen Spezifikationen der App. In Kapitel \ref{chap:functionalities} werden die Features beschrieben, die von der App in ihrer jeweils aktuellen Version angeboten werden. Und Kapitel \ref{} beinhaltet eine Dokumentation des Codes.
	
	\chapter{Funktionalitäten}\label{chap:functionalities}
		Dieses Kapitel beschreibt die Funktionalitäten, die die App in ihrer jeweils aktuellen Version anbietet. Dabei sollen die jeweiligen funktionalen sowie nicht-funktionalen Anforderungen aufgezeigt werden und nicht die programmiertechnische Umsetzung, die dann in Kapitel \ref{} folgt.
		
		\section{Campus-Navigation}\label{sec:campus_navigation}
		
		\section{Stundenplan}\label{sec:timetable}
		
		\section{Variabel einsetzbare, moderierbare Aufgabenlisten}\label{sec:todo_list}

\end{document}