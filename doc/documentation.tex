\documentclass[noindent]{tudreport}
\usepackage[utf8]{inputenc}
\usepackage[german]{babel}
\usepackage{amsmath}
\usepackage{listings}
\title{TUDas - Organisationsapp der Technischen Universität Darmstadt}
\subtitle{Ergebnis des E-Learning Projektpraktikums WS18-19}
\subsubtitle{Benedikt Lins (1799381) und Stefan Thaut (1800351)\\
			Fachbereich 20 - Informatik\\
			\today}
\setlength{\parindent}{0em} 

\DeclareMathOperator{\start}{start}
\DeclareMathOperator{\getEnd}{end}
\DeclareMathOperator{\hour}{hour}

\begin{document}
	\lstset{language=Java}
	\maketitle
	\tableofcontents
	
	\chapter{Motivation}
		Jeder Student oder der, der einmal einer gewesen ist, kennt es: Du kommst das erste Mal in die Uni und hast keine Ahnung, was zu tun ist. Alles ist komplett neu und Du bist froh, wenn Du die Räume findest, in denen Du Dich laut Deinem Willkommensbrief einfinden sollst. Selbst wenn Du dann nach ein paar Tagen herausgefunden hast, welche Straßenbahn Dich in die Uni bringt, bist Du nach ein paar Wochen immer noch ratlos, welche Dokumente du gegebenenfalls wo nachreichen musst, wo Du Dich zu welchen Veranstaltungen und Prüfungen anmelden sollst und in welchem Kellerraum nun dieser Treffpunkt Mathe stattfindet.\\
		
		Diesem Problem soll sich die (Android-)App \textit{TUDas} widmen. Die App soll als Organisationsplattform für Studenten verschiedener Fachbereiche während ihres ersten Semesters dienen. Dabei soll TUDas vorhandene Plattformen, wie \textit{Moodle}\footnote{www.moodle.tu-darmstadt.de [zuletzt aufgerufen: 23.11.2018]} oder die \textit{OAPP} \footnote{www.oapp.tu-darmstadt.de [zuletzt aufgerufen: 23.11.2018]} unterstützen und nicht ersetzen.\\
		%TODO worin unterscheidet sich die App von OAPP und Moodle und Google Maps etc?
		
		Diese Dokumentation soll einerseits die Funktionalitäten der App TUDas festhalten und andererseits als Übersicht für zukünftige Entwickler dienen. Kapitel \ref{chap:technical_specifications} behandelt die technischen Spezifikationen der App. In Kapitel \ref{chap:functionalities} werden die Features beschrieben, die von der App in ihrer jeweils aktuellen Version angeboten werden. Und Kapitel \ref{chap:documentation} beinhaltet eine Dokumentation des Codes.
		
	\chapter{Technische Spezifikationen}\label{chap:technical_specifications}
		Aus Zeit- und Komplexitätsgründen wird die App zunächst nur für das \textit{Android}-Betriebssystem\footnote{https://www.android.com/ [zuletzt aufgerufen: 24.11.2018]} entwickelt.
	
	\chapter{Funktionalitäten}\label{chap:functionalities}
		Dieses Kapitel beschreibt die Funktionalitäten, die die App in ihrer jeweils aktuellen Version anbietet. Dabei sollen die jeweiligen funktionalen sowie nicht-funktionalen Anforderungen aufgezeigt werden und nicht die programmiertechnische Umsetzung, die dann in Kapitel \ref{chap:documentation} folgt.
		
		\section{Campus-Navigation}\label{sec:campus_navigation}
		
		\section{Stundenplan}\label{sec:timetable}
			Im Stundenplan werden die selbst erstellten Termine des Nutzers angezeigt. Hier können grundsätzlich zwei Typen von Terminen unterschieden werden:
			\begin{itemize}
				\item Einfache Termine
				
				\item Wiederholende Termine
			\end{itemize}
			Die Charakteristik der einfachen Termine impliziert eine einmalige Anzeige im Stundenplan an einem bestimmten Datum. Daher ist es sinnvoll, den Stundenplan datumsabhängig anzuzeigen. Da die Bildschirmbreite des Smartphones in der vertikalen Ausrichtung deutlich begrenzt ist, muss auch die Anzahl der angezeigten Tage beschränkt werden. Zwei Tage, das heißt also der aktuelle Tag und der darauffolgende Tag, scheint eine sinnvolle Wahl zu sein, da man an diesen Tagen gegenwärtig am meisten interessiert ist. Der ausgewählte Zeitraum wird in einer Kopfzeile über den Tageslisten angezeigt. Auf der linken Seite sind in vertikaler Ausrichtung die Uhrzeiten in stündlichem Abstand untereinander angeordnet. Dabei beginnt der Stundenplan oben mit der maximalen ganzen Stunde, die kleiner als oder gleich allen Anfangsuhrzeiten von Terminen im angezeigten Zeitraum ist und endet unten mit der minimalen ganzen Stunde, die größer als oder gleich allen Enduhrzeiten der gleichen Termine sind.\\\\
			Viele Nutzer verwenden üblicherweise eine andere Applikation zur Verwaltung ihres Kalenders. Daher soll TUDAS bei Einstellen eines persönlichen Termins zumindest für den Google-Kalender überprüfen, ob sich der Termin nicht mit einem bereits erstellten Termin überschneidet.
		
		\section{Variabel einsetzbare, moderierbare Aufgabenlisten}\label{sec:todo_list}
			Da Erstsemesterstudenten oft nicht wissen, welche Termine und Veranstaltung für sie wichtig sind, ist es sinnvoll, diese Termine bereitzustellen. Dies setzt voraus, dass solche Termine von einer externen Institution erstellt und editiert werden. Eine solche Funktionalität benötigt einen zentralen Speicherort, wo die erstellten Termine gespeichert und von der App abgerufen werden können.\\
			Auf der anderen Seite gibt es spezifische Termine, die nicht jeder Student angezeigt bekommen soll, um nicht mit Informationen überflutet zu werden. Hierfür soll eine Art Abonnement-System etabliert werden, mittels dem Studenten verschiedene Labels abonnieren können, unter denen Informationen zu dem gleichen Thema zusammengefasst werden. Diese Labels können vollkommen frei gewählt werden und können beispielsweise für verschiedene Lehrveranstaltungen (z.B. eine Vorlesung) aber auch für die Termine der Orientierungsphase stehen.\\
			Um solche Termine zu erstellen und zu bearbeiten, wird ein Webinterface mit einer zugrundeliegenden Datenbank verwendet, welches über einen herkömmlichen Browser erreichbar ist. Fachschaften und andere Institutionen benutzen oftmals schon eine Software für die Verwaltung ihrer Termine. Daher soll das Webinterface auch eine Importfunktion besitzen, die das Hochladen von Kalenderdateien ermöglicht und für verschiedene Dateiformate erweiterbar ist.
		
	\chapter{Dokumentation der Implementierung}\label{chap:documentation}
		\section{Stundenplan}
			Wie in Abschnitt \ref{sec:timetable} beschrieben, müssen grundlegend zwei Arten von Terminen unterschieden werden. Ein wiederholender Termin hat im Vergleich zu einem einfachen Termin zusätzlich ein Startdatum, an dem die Wiederholung des Termins beginnt und ein Enddatum, wann die Wiederholung endet. Notwendig ist ebenfalls eine Wiederholungsvorschrift. Möchte man nun eine Teilmenge von wiederholenden Terminen löschen, so ist es notwendig, dass die Termine nicht zur Laufzeit auf Grundlage der Wiederholungsvorschrift berechnet werden sondern die Termine schon gespeichert wurden. Ansonsten wäre es nicht möglich zu spezifizieren, welche Termine von der Wiederholungsvorschrift ausgeschlossen werden sollen. Die wiederholenden Termine unterscheiden sich dann nur noch in den Daten. Der Titel und die Beschreibung beispielsweise sind für alle wiederholenden Termine jeweils gleich. Um Redundanzen zu vermeiden, werden solche Attribute von den reinen Datumsangaben getrennt. Die Klasse \lstinline!AppointmentContent! beinhaltet die statischen Informationen einer Terminsammlung und die Klasse \lstinline!Appointment! enthält die Datumsangaben. Die Attribute der Klassen sind im UML-Diagramm in Abbildung \ref{fig:uml_appointment} zu sehen.\\
			Für eine mathematische Beschreibung des Projekts müssen diverse Hilfsfunktionen definiert werden. Sei $A$ die Menge aller Termine des Benutzers und $\mathbb{D}$ die Menge aller Datumsangaben, die auch den Zeitpunkt am Tag beinhalten. Die Funktion
			\begin{align}
\start: A \rightarrow \mathbb{D} \label{fun:start}
			\end{align}
			liefert für einen gegebenen Termin den Startzeitpunkt. Analog liefert die Funktion
			\begin{align}
\getEnd: A \rightarrow \mathbb{D} \label{fun:end}
			\end{align}
			den Endzeitpunkt des gegebenen Termins. Für eine bestimmte Datumsangabe gibt die Funktion
			\begin{align}
\hour: \mathbb{D} \rightarrow \{0, \dots, 23\} \label{fun:hour}
			\end{align}
			die Stunde der Angabe aus.
			
			\begin{figure}[h]
				\centering
				\includegraphics[scale=0.5]{img/uml_appointment.png}
				\caption{UML-Diagramm der Klassen \lstinline!AppointmentContent! und \lstinline!Appointment!}
				\label{fig:uml_appointment}
			\end{figure}
			
			\subsection{Modellierung der Anfangs- und Endzeitbedingung}
				Die in Abschnitt \ref{sec:timetable} beschriebene Anforderung, dass der Stundenplan mit der maximalen Stunde beginnen soll, die kleiner als die Startzeiten aller Termine im betrachteten Zeitraum ist, kann wie folgt formalisiert werden: Für ein gegebenes Start- und Enddatum $s \in \mathbb{D}$, bzw. $e \in \mathbb{D}$ ist
				\begin{align}
A_f = \{a \in A: s \leq a \leq e\}
				\end{align}
				die Menge aller Termine im gegebenen Zeitraum. Dann ist die gesuchte Stunde $h$, zu der der Stundenplan beginnen soll, mithilfe der Funktionen \ref{fun:start} und \ref{fun:hour} wie folgt definiert:
				\begin{align}
h_s = \max\{h_s \in \{0, \dots, 23\}: \forall a \in A_f: h_s \leq \hour(\start(a)) \}
				\end{align}
				Analog ist die Stunde der Endzeit $h_e$ definiert:
				\begin{align}
h_e = \min\{h_e \in \{0, \dots, 23\}: \forall a \in A_f: h_e \geq \hour(\getEnd(a)) \}
				\end{align}
			
			\subsection{Positionierung der Termine im Stundenplan}
				Die Termine sollen in einem tabellenartigen Format angezeigt werden. Dabei soll jede Spalte der Tabelle einen Tag repräsentieren. Innerhalb eines Tages sollen die Termine untereinander angezeigt werden. Programmtechnisch wird jeder Tag über ein Layout beschrieben, dem Elemente vertikal hinzugefügt werden können. Für Zeiten, zu denen kein Termin existiert, wird dem Layout ein leeres Feld hinzugefügt. Damit beschränkt sich die korrekte Positionierung der Termine auf das Berechnen der Höhen der einzelnen Elemente. Dazu wird eine Konstante $L$ eingeführt, die die Anzahl der Pixel pro Minute angibt. Die Dauer eines Termins wird dann in Minuten berechnet und mit der Konstante multipliziert, um die entsprechende Höhe zu berechnen. Für die Anzeige der Termine eines Tages sind also folgende Schritte notwendig:
				\begin{enumerate}
					\item Laden der Termine des Tages $d$ als geordnete Menge $A_d$ nach ihren Anfangszeitpunkten.
				\end{enumerate}
	
	\chapter{Codespezifische Dokumentation}
		In diesem Kapitel soll ein konkreter Überblick über und mit wichtigen Anmerkungen zu dem Code gegeben werden.
		
		\section{Übersicht}
			

\end{document}